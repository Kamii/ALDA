% Alla i gruppen ska lämna in individuellt i Moodle. 
% Det ska dock vara samma fil för alla så att vi kan se vilka som har jobbat ihop.
\documentclass[a4paper,10pt,oneside,onecolumn]{article}

\usepackage[swedish]{babel}

% Får ni problem med att pdf-filen ser suddig ut på skärmen så beror det på att ni inte har 
% en fullständig fontuppsättning och att bitmappade fonter i fel skala används. Den bästa lösningen 
% är att uppdatera fontuppsättningen, men man kan också plocka bort denna rad. Avstavning fungerar 
% dock betydligt sämre då.
\usepackage[T1]{fontenc}

\usepackage[latin1]{inputenc}
\usepackage{a4wide}
\usepackage{graphicx}
\usepackage{cite}
\usepackage{url}
\usepackage{listings}
\usepackage{textcomp}
\lstset{language=Java, tabsize=3, numbers=left, frame=line}

% Glöm inte att uppdatera rubriken

\title{ALDA VT12: Linjära datastrukturer samt introduktion till algoritmanalys}

% Sortera gruppdeltagarna efter Efternamn. Om två personer har samma efternamn sorteras dessa efter förnamnen.
% Detta är viktigt eftersom vissa uppgifter bestäms av i vilken ordning namnen står. Tar man fel uppgift vid någon av dessa tillfällen blir man automatiskt underkänd.
\author{Leon Hennings\\leonh \and Kamyar Sajjadi\\kamy-saj}

\begin{document}

\maketitle

\section{Lazy Deletion}
\lstinputlisting{SimpleLinkedList.java}

\section{Två stackar i en array}
\lstinputlisting{DoubleStack.java}

\section{Algoritmanalys}

Denna loop itererar n gånger när n=10 så blir summan 10. Det innebär att komplexiteten är linjär O(N).
Vid 10 iterationer tar det ca 600 nanosekunder. När vi ökar n till 100 tar det ca 1300 nanosekunder. 
\begin{lstlisting}
//Exempel 1
int sum = 0 ;
for(int i = 0; i<n ;i++)
	sum++;
\end{lstlisting}

Exempel 2 kommer att ökas kvadratiskt O(N2). Eftersom det är två loopar som har O(N) dvs är linjära så blir tidskomplexiteten N * N = N2.
För varje varv i den övre loopen kommer den undre loopen gå n gånger. Vid mätning med nanosekunder ger n=10 ca 2900 nanosekunder samt n=100 tar det ca 160000 nanosekunder. 

\begin{lstlisting}
//Exempel 2
int sum = 0 ;
for (int i = 0 ; i<n ; i++)
	for (int j = 0; j<n; j++)
		sum++;
\end{lstlisting}

\begin{lstlisting}
//Exempel 3
int sum = 0 ;
for (int i=0; i<n; i++)
	for (j=0; j<n*n; j++)
		sum++;
\end{lstlisting}

\begin{lstlisting}
//Exempel 4
int sum = 0 ;
for (int i = 0; i<n; i++)
	for (int j = 0; j<i ; j++)
		sum++;
\end{lstlisting}

\begin{lstlisting}
//Exempel 5
int sum = 0 ;
for (int i=0; i<n ; i++)
	for (int j=0; j<i*i ; j++)
		for (int k=0; k<j; k++)
			sum++;
\end{lstlisting}

\begin{lstlisting}
//Exempel 6
int sum = 0 ;
for (int i=1; i<n; i++)
	for (int j=1; j<i*i ; j++)
		if ( j % i == 0 )
			for (int k=0; k<j; k++)
				sum++;
\end{lstlisting}
%\lstinputlisting{SimpleLinkedList.java}

%\lstinputlisting[firstline=3, lastline=5]{SimpleLinkedList.java}

% För att referena till en bild skriver man ''se figur \ref{fig:exempelbild}''.

\begin{figure}[htbp]
	\centering
%		\includegraphics[width=1.00\textwidth]{bild}
	\caption{exempelbild}
	%\label{fig:exempelbild}
\end{figure}

% Nedanstående är bara för om ni vill använda bibtex för att hantera referenser. 

%\bibliographystyle{plain}
%\bibliography{bibtex}
%\bibdata{bibtex}


\end{document}
