% Alla i gruppen ska l�mna in individuellt i Moodle. 
% Det ska dock vara samma fil f�r alla s� att vi kan se vilka som har jobbat ihop.
\documentclass[a4paper,10pt,oneside,onecolumn]{article}

\usepackage[swedish]{babel}

% F�r ni problem med att pdf-filen ser suddig ut p� sk�rmen s� beror det p� att ni inte har 
% en fullst�ndig fontupps�ttning och att bitmappade fonter i fel skala anv�nds. Den b�sta l�sningen 
% �r att uppdatera fontupps�ttningen, men man kan ocks� plocka bort denna rad. Avstavning fungerar 
% dock betydligt s�mre d�.
\usepackage[T1]{fontenc}

\usepackage[latin1]{inputenc}
\usepackage{a4wide}
\usepackage{graphicx}
\usepackage{cite}
\usepackage{url}
\usepackage{listings}
\usepackage{textcomp}
\usepackage{color}
\usepackage{hyperref}
\hypersetup{linktocpage}

\definecolor{javared}{rgb}{0.6,0,0} % for strings
\definecolor{javagreen}{rgb}{0.25,0.5,0.35} % comments
\definecolor{javapurple}{rgb}{0.5,0,0.35} % keywords
\definecolor{javadocblue}{rgb}{0.25,0.35,0.75} % javadoc

\lstset{language=Java,
basicstyle=\ttfamily,
keywordstyle=\color{javapurple}\bfseries,
stringstyle=\color{javared},
commentstyle=\color{javagreen},
morecomment=[s][\color{javadocblue}]{/**}{*/},
numbers=left,
numberstyle=\small\color{black},
stepnumber=1,
numbersep=10pt,
tabsize=2,
showspaces=false,
showstringspaces=false}

% Gl�m inte att uppdatera rubriken
\title{ALDA VT12:Hashning och prioritetsk�er}

\author{Leon Hennings\\leonh \and Kamyar Sajjadi\\kamy-saj}

\begin{document}

\maketitle
\tableofcontents
\paragraph{}
Koden finns att ladda ner p� \url{http://people.dsv.su.se/~kamy-saj/download/ALDA/}

Detta f�r att det ska bli l�ttare f�r er att g�ra en peer-review i er favorit editor. 

\begin{center}
Koden till zip-filen �r: \textbf{FgH9sg6}
\end{center}
\newpage


\section{MyMiniHeap - Konstruktor och privata variabler}
\begin{lstlisting}
 /**
 * MyMiniHeap is a d-heap
 */
public class MyMiniHeap<T extends Comparable<? super T>> implements MiniHeap<T>
{
  private T[] heap;
  private int size;
  private int d;
  private static final int DEFAULT_CAPACITY = 10;

  /**
   * Constructor with no arguments.
   * This constructor will call the constructor with 2 arguments. 
   * The standard size of the d-heap i 10 with 2 childs 
   */
  public MyMiniHeap()
  {
    this(2,DEFAULT_CAPACITY);
  }

  /**
   * Constructor with one arguments.
   * @param childs The number of childs
   * Creates a d-heap with standard capacity of 10 and number
   * of childs is the argument. This constructor will call the 
   * constructor with 2 arguments. 
   */
  public MyMiniHeap(int childs)
  {
    this(childs,DEFAULT_CAPACITY);
  }

  /**
   * Constructor with two arguments.
   * @param childs The number of childs.
   * @param capacity The size of the heap. 
   * @throws IllegalStateException If childs is less then 2.
   * Creates a d-heap with given capacity and childrens
   */
  @SuppressWarnings("unchecked")
  public MyMiniHeap(int childs, int capacity)
  {
    if(childs < 2)
      throw new IllegalArgumentException();

    size = 0;
    d = childs;
    heap = (T[]) new Comparable[capacity+1];
  }
\end{lstlisting}
\newpage

\section{MyMiniHeap - }
\begin{lstlisting}

\end{lstlisting}
\newpage


\section{MyMiniHeap - }
\begin{lstlisting}

\end{lstlisting}
\newpage

\section{MyMiniHeap - }
\begin{lstlisting}
                          
\end{lstlisting}
\newpage

\section{MyMiniHeap - }
\begin{lstlisting}
                                                                                           
\end{lstlisting}
\newpage


\section{Hashing}


\newpage
\section{Poolfr�ga}
\subsection{Fr�ga 1}

\subsection{Fr�ga 2}

\subsection{Fr�ga 3}

%\lstinputlisting[firstline=3, lastline=5]{SimpleLinkedList.java}

% Nedanst�ende �r bara f�r om ni vill anv�nda bibtex f�r att hantera referenser. 

%\bibliographystyle{plain}
%\bibliography{bibtex}
%\bibdata{bibtex}


\end{document}
